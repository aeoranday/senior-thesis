\documentclass[thesis.tex]{subfiles}

\begin{document}
Dark matter constitutes 85\% of the matter in our Universe and was made evident by observations on galaxy formation, gravitational lensing, and the cosmic microwave background \cite{DM_Hist}.
However, detecting dark matter experimentally is exceedingly difficult with particle physics detectors.
These particles are suspected to be weakly interacting \cite{1TDM_DataAnalysis} such that any detection attempt would need to be sensitive to recoils at keV levels of energy.

\par The XENON1T detector operated as the most sensitive dark matter detector \cite{Xenon1t}, and the soon-to-be-active XENONnT detector plans to overtake that title \cite{nT_Projection}.
In these detectors, there are two important elements that need to be reconstructed: energy and position.
By reconstructing these key elements, we are able to accept or reject numerous observations if the reconstructed position is within the detector's fiducial volume and if the reconstructed energy is within a rejection threshold \cite{1TDM_DataAnalysis}.

\par Previous machine learning implementations for position reconstruction perform well enough \cite{Bart}, but still run into issues with reconstruction outside the bounds of the detector as well as having an inward reconstruction bias.
Finding the type of machine learning algorithm that is the most appropriate for these problems is essential for the best use of the detector's fiducial volume.
This project produced the first application of graph convolutional neural networks (GCNNs) for position reconstruction in the dark-matter field, and one of the first applications of a GCNN for use in regression.
\end{document}
